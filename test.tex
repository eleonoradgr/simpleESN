In questo capitolo viene testata la rete neurale implementata attraverso il task NARMA e vengono confrontati i risultati ottenuti con quelli della letteratura.

\section{$\mathbf{10^{th}}$ order NARMA system}
Questo task consiste nella predizione di un sistema di $10^{th}$ ordine di media mobile autoregressivo non lineare. Il task è stato introdotto in Atiya e Palos (2000) ed è stato trattato con le ESN in Jaeger (2002) e in Cernansk\'y e Ti\v{n}o (2008). L'input del sistema è una sequenza di elementi $\mathbf{u}(n)$ scelti secondo una distribuzione uniforme in $[0,0.5]$. L'output del sistema è calcolato come:

\begin{equation}\label{narma}
	\bar{\mathbf{y}}(n) = 0.3\bar{\mathbf{y}}(n-1) + 0.05\bar{\mathbf{y}}(n-1)\biggl( \sum_{i=1}^{10}\bar{\mathbf{y}}(n - i)\biggr) + 1.5 \mathbf{u}(n-10)\mathbf{u}(n-1) +0.1 .
\end{equation}

Dato il valore di input $\mathbf{u}(n)$, il task è di predire il corrispondente valore di $\bar{\mathbf{y}}(n)$. Il training set è formato da $N_{train}=2200$ esempi input-target,dei quali $N_{transient}=200$ sono di transient iniziale. Una sequenza di lunghezza $N_{test}=2000$ viene usata per il test.\\
Il task viene generato attraverso la funzione \textit{genetateTask()}, passando come argomento \textit{Tasks.narma}, questa funzione permette di generare tutti i task che sono elencati nell'enumerazione \textit{Tasks}, questi ultimi vengono creati secondo dei criteri ben precisi in modo da poter fornire alla rete tutte le informazioni necessarie. Per verificare che un task abbia tutti i campi necessari per essere ben definito è utilizzata la funzione \textit{isTask()}.

\section{ESNtrain() ed ESNtest()}
Per facilitare l'uso della rete sviluppata viene fornita una funzione chiamata \textit{ESNtrain()}. La funzione si occupa dell'inizializzazione delle matrici dei pesi, del calcolo dello stato del reservoir e del training del readout invocando le funzioni illustrate nel capitolo \ref{cap:test}. A questa funzione devono essere passati come argomenti un task valido e tutti i parametri che si intendono usare nella rete, qualora questi non fossero presenti vengono utilizzati dei parametri standard illustrati nella tabella sottostante.
\begin{table}[h]
	\begin{center}
	 \begin{tabular}{|l|c|r|}
	 	\hline
	 	\textbf{parametri}		&\textbf{nome parametro}&   \textbf{valore default}\\
	 	\hline
	 	input scaling			&  \textit{scale\_in}	&   0.1\\
	 	\hline
	 	unità reservoir 		&  \textit{nr}          &   100\\
	 	\hline
	 	tipo di distribuzione	&  \textit{dist}        &   'ud'\\
	 	\hline
	 	densità connessione		&  \textit{density\_con}&   1\\
	 	\hline
	 	raggio spettrale		&  \textit{rho}			&   0.9\\
	 	\hline
	 	leaky integrator		&  \textit{alpha}		&   1\\
	 	\hline
	 	input bias				&  \textit{bias}		&   1\\
	 	\hline
	 	par regolazizzazione	&  \textit{lambda}		&   0\\
	 	\hline
	 	misura errore			&  \textit{error}		&   'mse'\\
	 	\hline
	 	risultati test			&  \textit{test}		&   true\\
	 	\hline
	 \end{tabular}
	\end{center}
	\caption{Parametri default \textit{ESNtrain()}}
\end{table}

In particolare si ha il parametro \textit{test} che se settato a true permette di ottenere anche il risultato di test,eseguito dopo il training. Se viene usata questa opzione bisogna far attenzione a non usare il valore di test per la scelta del modello, in questo caso infatti la \textit{model selection} verrebbe compromessa. Per il test della rete deve essere usata la funzione ESNtest(). Questa funzione prende in ingresso il task ed eventualmente altri parametri per la realizzazione della rete, infatti si può decidere se passare o meno come parametri un reservoir ed un readout allenato. Nel primo caso viene eseguita solo una fase di test, altrimenti viene inizializzata un'altra rete secondo i parametri passati come argomento; quest'ultima operazione assume un certo valore soprattutto nel caso in cui i parametri passati come argomento sono quelli di una model selection precedente.

\section{Risultati}
Per poter ottenere dei risultati confrontabili con quelli in \cite{Markovianfactor:paper} viene fatta una model selection al variare del parametro \textit{rho}, ovvero del raggio spettrale. Nelle figure successive sono mostati in ordine i valori dell'errore di train, dell'errore di test ottenuto con la funzione \textit{ESNtrain()}, di quello di test ottenuto con la funzione \textit{EStest()} avente come parametri rispettivamente quelli ottenuti della model selection ed il reservoir ed il readout prodotti in precedenza.

Si può notare come l'errore diminuisca al crescere del valore del raggio spettrale, andamento aspettato nel caso di task non lineare come questo.



\section{Conclusioni}



