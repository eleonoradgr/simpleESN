\pagenumbering{arabic}

Il Reservoir Computing \`e un metodo di grande interesse nel campo della ricerca sulle reti neurali ricorrenti, grazie alla sua semplicit\`a e alle sue buone performance su una certa variet\`a di task. Con Reservoir Computing ci si riferisce  ad una classe di modelli di reti neurali ricorrenti che sono caratterizzati da una parte dinamica ricorrente e una parte di output semplice. L'idea alla base del Reservoir Computing \`e quella di prendere una rete ricorrente random di semplici nodi, ad esempio neuroni sigmoidali,chiamata appunto reservoir. I nodi sono collegati tra loro, associando dei pesi ai collegamenti che possono essere riscalati per imporre il regime dinamico desiderato. Il reservoir permette dunque di mappare un input in uno spazio di dimensione maggiore che nel caso di task di classificazione pu\'o aumentare la probabilit\'a di separazione lineare tra i dati. Oltre  a questo permette anche di tenere traccia della storia degli input nel tempo, questi aspetti lo rendono ideale per molti task interessanti nel mondo reale che richiedono elaborazione temporale e mapping non lineare.\\
 È proprio in questo contesto che si collocano le Echo State Network, affermandosi per la loro praticità e per essere concettualmente semplici. Per ottenere delle buone performance sui task vanno però tenuti in considerazione diversi aspetti e proprietà di cui si parlerà in seguito. Questa semplice implementazione delle Echo State Network vuole essere un primo approccio guidato alla conoscenza dei meccanismi sui cui si basa questo modello e alla loro applicazione.
In appendice è riportato il codice matlab della ESN sviluppata.


