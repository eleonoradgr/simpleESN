\pagenumbering{arabic}

Il \textit{Reservoir Computing} (RC) \`e un metodo di grande interesse nel campo della ricerca sulle reti neurali ricorrenti (RNR), grazie alla sua semplicit\`a e alle sue buone performance su una certa variet\`a di \textit{task}. Con \textit{Reservoir Computing} ci si riferisce  ad una classe di modelli di reti neurali ricorrenti che sono caratterizzati da una parte dinamica ricorrente e una parte di output semplice. L'idea alla base del \textit{Reservoir Computing} \`e quella di prendere una rete ricorrente random di semplici nodi, ad esempio neuroni sigmoidali,chiamata appunto \textit{reservoir}. I nodi sono collegati tra loro, associando dei pesi ai collegamenti che possono essere riscalati per imporre il regime dinamico desiderato. Il \textit{reservoir} permette dunque di mappare un input in uno spazio di dimensione maggiore che nel caso di \textit{task} di classificazione pu\'o aumentare la probabilit\'a di separazione lineare tra i dati. Oltre  a questo permette anche di tenere traccia della storia degli input nel tempo, questi aspetti lo rendono ideale per molti \textit{task} interessanti nel mondo reale che richiedono elaborazione temporale e \textit{mapping} non lineare.\\
 È proprio in questo contesto che si collocano le \textit{Echo State Network} (ESN), affermandosi per la loro praticità , per essere concettualmente semplici e per l'efficienza in fase di allenamento dovuta al fatto che non vengono addestrati i pesi delle connessioni ricorrenti. Per ottenere delle buone performance sui \textit{task} vanno però tenuti in considerazione diversi aspetti e proprietà di cui si parlerà in seguito. Questa semplice implementazione delle \textit{Echo State Network} vuole essere un primo approccio guidato alla conoscenza dei meccanismi sui cui si basa questo modello e alla loro applicazione.
In appendice è riportato il codice matlab della ESN sviluppata.


